%%%%%%%%%%%%%%%%%%%%%%%%%%%%%%%%%%%%%%%%%
% 模板資訊:
% 模板名稱:Ernie's Article
% 版本:1.0 (2023.03.26)
% 作者:莊程翔 Ernie Cheng-Xiang Zhuang
% 編譯器:XeLaTeX -> BibTeX -> XeLaTeX -> XeLaTeX
%
% 製作本模板之目的:
% 由於,LaTeX 對初學者來說不太友善,且 Word 在於數學式及中文排版上不大美觀。因此我製作了這個模板,同時附上清楚明瞭的註解、許多常用且好用的封包,以及客製化設定指令,以便 LaTeX 初學者能夠輕鬆地完成作業、報告甚至是學位論文。
% 如果您有任何問題,可以透過以下兩種方式聯繫我: 
% 1. 網站:https://www.ernie-zhuang.com/contact
% 2. Email:erniezhuang1127@gmail.com
%%%%%%%%%%%%%%%%%%%%%%%%%%%%%%%%%%%%%%%%%

%----------------------------------------------------------------------------------------
%	封包與文檔配置
%----------------------------------------------------------------------------------------

% 輸出的字體是以 Type 1 編碼
\usepackage[T1]{fontenc} 

% 自訂字體的封包
\usepackage{fontspec} 

%% 設定英文字體
\setmainfont{Times New Roman} 

%% 設定中文字體
\usepackage{xeCJK}
\setCJKmainfont[AutoFakeBold=3.5 , AutoFakeSlant=0.2, Path = fonts/]{源雲明體.ttf}
\setCJKmonofont[AutoFakeBold=3.5 , AutoFakeSlant=0.2, Path = fonts/]{源雲明體.ttf}
\XeTeXlinebreaklocale "zh"
\XeTeXlinebreakskip = 0pt plus 1pt

% 設定為 A4 紙的大小及上下左右的邊界
\usepackage[left=3.18cm, right=3.18cm, top=2.54cm, bottom=2.54cm]{geometry} 

% 行距,搭配後面得\begin{spacing}{1.5} 及 \end{spacing}
\usepackage{setspace} 

% 每段首行文字空行的封包
\usepackage{indentfirst} 

% 設定空 2 個字元 (2em)
\setlength{\parindent}{2em} 

% 呼叫任意大小字體的封包 (如 13.5pt)
\usepackage{type1cm} 

% 設置節的編號形式 (若不想變更節的樣式,將此以下至 \usepackage{color} 前都以 % 註釋掉。)
\usepackage{titletoc} % 調整節標題的形式
\usepackage[small]{titlesec} % 修改章節字型與大小。(可於中括號中加入 small, sf)
%\usepackage{zhnumber} % 編號為一、二⋯
%\renewcommand\thesection{\zhnum{section}、} % 調整節的編號 (還可以用 \Alph, \alph, Roman, roman)
%\renewcommand\thesubsection{\arabic{subsection}.} % 調整子節的編號
%\renewcommand\thesubsubsection{(\arabic{subsubsection})} % 調整小節的編號
%\makeatletter

% 設置章節編號與章節標題的距離
%\renewcommand\@seccntformat[1]{
%   {\csname the#1\endcsname}\hspace{0em}
%}

% 自訂字體顏色的封包
\usepackage{color} 

%% 自訂顏色
\definecolor{NTHU_Purple}{RGB}{126,35,138}
\definecolor{Default_Blue}{RGB}{52,51,171}

% 數學工具及符號
\usepackage{mathtools, amsmath, amsfonts, amsthm, latexsym} 

% 分別將數學符號間的間隔加大及加粗
\usepackage{newtxtext,newtxmath}

\theoremstyle{definition} % 數學定理及定義的風格,plain, remark, definition, thmsty
\newtheorem{Assum}{\textbf{假設}}
\newtheorem{Axiom}{\textbf{公理}}
\newtheorem{Def}{\textbf{定義}}
\newtheorem{Thm}{\textbf{定理}}
\newtheorem{Lemma}{\textbf{引理}}
\newtheorem{Corol}{\textbf{推論}}
\newtheorem{Property}{\textbf{性質}}
\newtheorem{Proposition}{\textbf{命題}}
\newtheorem{Claim}{\textbf{宣稱}}
\newtheorem{Remark}{\textbf{備註}}
\newtheorem{Note}{\textbf{註記}}
\renewcommand{\proofname}{\rm\textbf{證明}}
%\renewcommand{\qedsymbol}{}

% 打勾 \Checkmark、打叉 \XSolid
\usepackage{bbding} 

% 圖表自動編號的封包
\usepackage[justification=centering]{caption} 
\usepackage[justification=centering, format=hang]{subcaption}

%% 設定圖表編號及標籤的字體大小及字形
\captionsetup[figure]{font=normalsize, labelfont=md}
\captionsetup[table]{font=normalsize, labelfont=md}

% 導入圖形與表格的封包
\usepackage{graphicx}  % \scalebox{} 可用於將過大的表格縮小
\usepackage{booktabs}

% 允許表格的一格能多列呈現的封包
\usepackage{multirow} 

% 可指定表格排版的封包
\usepackage{array}

% 逆時針翻轉表格 90 度
\usepackage{lscape} 

% 長表格
\usepackage{longtable} 

% 表格內數字,以小數點或逗點對齊
\usepackage{dcolumn} 

% 調整圖的位置
\renewcommand{\textfraction}{0.15}
\renewcommand{\topfraction}{0.85}
\renewcommand{\bottomfraction}{0.85}
\renewcommand{\floatpagefraction}{0.60}

% 調整表中數字的對齊方式
\newcolumntype{d}[1]{D{,}{,}{#1}} % 以逗點對齊,應在排表格時鍵入 d{?}
\newcolumntype{.}[1]{D{.}{.}{#1}} % 以小數點對齊,應在排表格時鍵入 .{?}

% 文繞圖
\usepackage{wrapfig} 

% 增加註腳的間隔
\setlength{\footnotesep}{1em} 

% 超連結的封包
\usepackage{url}
\usepackage[colorlinks, bookmarks = false]{hyperref}

%% 設定各種超連結的顏色
\hypersetup{
	linkcolor = black,
	citecolor = blue,
	filecolor = blue,
	urlcolor = blue
} 

% 序列標號
\usepackage{enumerate} 

% 註釋掉大部分的封包
\usepackage{comment}

% 導入 pdf 檔的封包 
\usepackage{pdfpages} % \includepdf[pages=-]{檔名}

% 摘要的封包
\usepackage{abstract} 
\renewcommand{\abstractnamefont}{\Large} % 摘要,兩字的大小

% 附錄的封包
\usepackage{appendix} 

% 引注參考文獻的封包
\usepackage[sort]{natbib}

% 設定手動引注參考文獻的格式
\newcommand\laref{\bigskip\noindent\hangindent=1em} 

% 修改頁眉與頁足
%\usepackage{fancyhdr} 
%\pagestyle{fancy} % 設定頁眉與頁足的顯示風格,有 empty、plain、headings、myheadings、fancy。
%\renewcommand{\sectionmark}[1]{\markright{\thesection#1}} 
%\fancyhead[LO, LE]{}
%\fancyhead[RO, RE]{\rightmark}
%\addtolength{\headheight}{6pt}
%\renewcommand{\footrulewidth}{0pt}
%\extrarowheight=2pt

% 設定中文的標籤
\renewcommand{\figurename}{圖} 
\renewcommand{\tablename}{表} 
\renewcommand{\abstractname}{\textbf{摘要}} 
\renewcommand{\contentsname}{\textbf{目錄}}
\renewcommand{\listfigurename}{\textbf{圖目錄}}
\renewcommand{\listtablename}{\textbf{表目錄}}
\renewcommand{\refname}{\textbf{參考文獻}}
\renewcommand{\appendixpagename}{\Large \textbf{附錄}}
